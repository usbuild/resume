\documentclass[margin]{res}

\usepackage{fontspec,xltxtra,xunicode}
\usepackage{latexsym}
\usepackage{xeCJK}

\defaultfontfeatures{Mapping=tex-text}
\setCJKmainfont[BoldFont=SimHei,ItalicFont=KaiTi]{SimSun}
\setmainfont{Palatino Linotype}

\textwidth=5.2in % increase textwidth to get smaller right margin

\usepackage[hidelinks]{hyperref}
\begin{document} 

\address{ 张启超 \\ }
\address{ 南京大学, 汉口路22号, 210093 \\ (+86) 152-9838-8316  \\ \href{http://lecoding.com}{http://lecoding.com} \\ 
\href{mailto:njuzhangqichao@gmail.com}{njuzhangqichao@gmail.com} }

 
\begin{resume} 
 
\section{技能} 
   \begin{tabular}{l p{3in}}
      \bf{系统} &  Ubuntu, ArchLinux, Gentoo, CentOS \\
      \bf{编程语言} &  C, Java, PHP, JavaScript, C++, Python, Bash \\
      \bf{工具} & Vim, Git, GDB, IntelliJ, Eclipse, Maven \\
      \bf{框架} & jQuery, Yii, Backbone, Spring, Spring MVC
 \end{tabular}

\section{经验} 
{\bf 江苏中正信息科技有限公司} \hfill 2012-7-20 至 2012-12-20 
\begin{itemize} \itemsep -2pt 
\item 这是一个创业团队,主要使用 Yii 框架开发 PHP 程序。
\item 本人的工作主要是设计开放平台 OAuth2.0 认证和平台 API 。除此还进行一些前端的开发工作(基于 jQuery 和 Bootstrap )。
\item 在开发过程中加深了对前后端协作和团队开发的认识。
\end{itemize}

\section{项目}
 {\bf UWS,} 轻量级HTTP服务器\hfill 2012 - 2013
 \begin{itemize} \itemsep -2pt

\item 个人项目,使用C语言开发,旨在设计一个高性能的Web服务器,功能类似于 Nginx,截至目前版本已达到 0.0.10。
\item 功能包括:虚拟主机,自定义错误页,304 状态,gzip 和 deflate 压缩,http basic 认证,url 重写,简单的反向代理,FastCGI 支持。
\item 在虚拟机512M的内存情况下,ab 测试 1000 并发单进程处理量超过 7kqps
\item 为了解决服务器内存泄漏问题,本人编写了一个内存管理程序 usmem,借鉴了 Loki 的实现,使内存分配效率得到提高。
\item 目前测试站点已上线,\href{http://demo.lecoding.com}{http://demo.lecoding.com}
\end{itemize}

 
 
{\bf DistMEM,} NoSQL 数据库\hfill  2012 - 2013
\begin{itemize} \itemsep -2pt
\item 个人项目,使用C++开发。提供了简单的 API ,支持 set, get 和 del 操作。
\item 数据类型支持 string,float,list,int,其中 string 是二进制安全的。
\item 数据存储参照了 fat12 文件格式,索引使用了 B 树,为了提高访问性能采用了 LRU 作为缓存算法。
\item 数据交换协议参照了 Redis,并使用 Python 编写了客户端控制台。
\item 另外本人还使用 C 语言开发了php\_distmem 作为 PHP 语言访问该数据库的扩展。
\end{itemize}

{\bf 南大文集,} 南京大学文学爱好者社区 \hfill 2012
\begin{itemize} \itemsep -2pt
\item 合作项目,基于 Yii 框架。这是与学长合作的一个项目,旨在为南京大学文学爱好者提供一个单纯的文学交流平台。本人的工作有:
\item 社交网络分享模块和评论模块
\item 类 markdown 语法解析器和在 textarea 中显示下拉框的 jQuery 插件。
\item 其它一些前端和后端琐碎工作,为别人提供 RESTful 接口,修改页面CSS等。
\end{itemize}

{\bf UPLEX,} Python 语言的 Lex 解析器 \hfill 2012
\begin{itemize} \itemsep -2pt
\item 个人项目,使用Python语言开发的LEX程序。
\item 依据《编译原理》中的知识设计的 Python 版本的Lex,核心是正则表达式引擎。可对简单的正则使用 NFA->DFA->mDFA 过程处理得到最小化 DFA ,再依据 *.l 文件生成规则解析文件。
\end{itemize}

{\bf Lily-PHP-SDK,} 南京大学小百合论坛 SDK \hfill 2011 - 2012
\begin{itemize} \itemsep -2pt
\item 个人项目,使用PHP开发。为了封装南京大学小百合论坛的常用操作编写的 API。
\item 主要工作是解析页面DOM,采用了正则匹配,最终提供合乎规范的调用接口。
\item LilyMe 是该 SDK 的衍生项目,基于 jQuery Mobile 的小百合移动网页版。
\item LilyClient 是该 SDK 的衍生项目,这是一个 Android 版本的小百合客户端,添加了的部分功能如“@”、”位置标记”和“涂鸦”等。
\end{itemize}

{\bf 其它项目}\hfill 2011 - 2013
\begin{itemize} \itemsep -2pt
\item 轻博客,基于 Yii 和 jQuery 的博客系统
\item 网络白板,基于 Java 和 Swing 的远程白板程序,本人主要负责整体架构、语音传输和IPC功能。
\item 基于 Google App Engine 和 Python 的一些实用程序,如记事本,匿名网盘等。\href{http://tools.lecoding.com}{http://tools.lecoding.com}
\item 教务网自动选课插件。这是一个 Chrome 插件,用于南京大学教务网抢课,使用 JS 开发,实战效果显著。
\end{itemize}

\section{在校成绩} 
   \begin{tabular}{l p{3in}}
      GPA &  4.1/5.0 \\
      四 / 六级 & 617 / 546 \\
      获人民三等奖学金
 \end{tabular}

\section{相关链接}
   \begin{tabular}{l p{3in}}
     \underline{个人博客:} & \href{http://lecoding.com}{http://lecoding.com}\\
     \underline{GitHub:} &  \href{https://github.com/usbuild}{https://github.com/usbuild}\\
     \underline{Twitter:} & \href{https://twitter.com/usbuild}{https://twitter.com/usbuild}
 \end{tabular}

\end{resume} 
\end{document} 



